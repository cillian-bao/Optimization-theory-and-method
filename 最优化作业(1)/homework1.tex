\documentclass{article}
% 这里是导言区
%\usepackage{indentfirst}%缩进控制
\usepackage{listings}%插入代码
%\usepackage{mcode}
%ctex能够保证能够渲染英文
\usepackage{ctex}
\usepackage{textcomp}
\usepackage{graphicx}%插入图像
\usepackage{epstopdf}
\usepackage{amsmath}
\usepackage{graphicx}
\usepackage{subfigure}
\usepackage{geometry}%设置页边距
\usepackage{amssymb}
\usepackage{float}
\usepackage{xcolor}%定义了一些颜色  
\usepackage{colortbl,booktabs}%第二个包定义了几个*rule 
\usepackage[level]{datetime} 
\makeatletter
\newcommand{\rmnum}[1]{\romannumeral #1}
\newcommand{\Rmnum}[1]{\expandafter\@slowromancap\romannumeral #1@}
\makeatother
% \renewcommand\thesection{\roman{subsection}}
%\newdateformat{ukdate}{\ordinaldate{\THEDAY} \monthname[\THEMONTH]

\geometry{a4paper,scale=0.75}


\lstset{
tabsize=4, %tab 空格数
frame=shadowbox, %把代码用带有阴影的框圈起来
rulesepcolor=\color{red!20!green!20!blue!20}, %代码块边框为淡青色
keywordstyle=\color{blue!90}\bfseries, %代码关键字的颜色为蓝色, 粗体
showstringspaces=false, %不显示代码字符串中间的空格标记
stringstyle=\ttfamily, %代码字符串的特殊格式
keepspaces=true, %
breakindent=22pt, %
numbers=left, %左侧显示行号
stepnumber=1, %
numberstyle=\tiny, %行号字体用小号
basicstyle=\footnotesize, %
showspaces=false, %
flexiblecolumns=true, %
breaklines=true, %对过长的代码自动换行
breakautoindent=true, %
breakindent=4em, %
aboveskip=1em, %代码块边框
}

\title{作业(1)}
\author{31202008881        \quad \quad \quad
          鲍泽安}

\begin{document}
\setlength{\parindent}{2em}
\maketitle
作业(1):\\
(1)(2)单纯形法
(1):
\[
\begin{split}
min \quad -9x_1-16x_2\\
\begin{aligned}
s.t.\quad &x_1+4x_2+x_3=80,\\
     &2x_1+3x_2+x_4=90,\\
     &x_j \geq 0,j=1,\cdots,4.\\
\end{aligned}
\end{split}
\]
(2):
\[
\begin{split}
max \quad x_1+3x_2\\
\begin{aligned}
s.t.\quad &2x_1+3x_2+x_3=6,\\
     &-x_1+x_2+x_4=1,\\
     &x_j \geq 0,j=1,\cdots,4.\\
\end{aligned}
\end{split}
\]
解
(1):利用单纯性表\\
\begin{table}[h]
    \setlength{\belowcaptionskip}{0.cm}
    \centering
    \begin{tabular}
        {>{}rccccc}
        \toprule[1pt]
        \rowcolor[gray]{0.9}    &$x_1$ &$x_2$   &$x_3$  &$x_4$ & \\
        \midrule
        $x_3$   &1   &\multicolumn{1}{>{\columncolor{green}[0pt][0pt]}c}{4}  &1  &0 & 80  \\
        $x_4$   &2 &3    &0  &1 & 90  \\
        & 9 & 16 & 0 & 0 & 0\\
        \bottomrule[1pt]
        \end{tabular}
\end{table}
\\
第二次迭代
\begin{table}[h]
    \setlength{\belowcaptionskip}{0.cm}
    \centering
    \begin{tabular}
        {>{}rccccc}
        \toprule[1pt]
        \rowcolor[gray]{0.9}    &$x_1$ &$x_2$   &$x_3$  &$x_4$ & \\
        \midrule
        $x_2$   &$\frac{1}{4}$  &1  &$\frac{1}{4}$  &0 & 20  \\
        $x_4$   &\multicolumn{1}{>{\columncolor{green}[0pt][0pt]}c}{$\frac{5}{4}$} &0    &-$\frac{3}{4}$  &1 & 30  \\
        & 5 & 0 & -4 & 0 & -320\\
        \bottomrule[1pt]
        \end{tabular}
\end{table}
\\
\\
\\
第三次迭代
\begin{table}[h]
    \setlength{\abovecaptionskip}{0.cm}
    \centering
    \begin{tabular}
        {>{}rccccc}
        \toprule[1pt]
        \rowcolor[gray]{0.9}    &$x_1$ &$x_2$   &$x_3$  &$x_4$ & \\
        \midrule
        $x_2$   &0   &1  &$\frac{2}{5}$  &$-\frac{1}{5}$ & 14  \\
        $x_1$   &1 &0    &$-\frac{3}{5}$  &$\frac{4}{5}$ & 24  \\
        & 0 & 0 & -1 & -4 & -440\\
        \bottomrule[1pt]
        \end{tabular}
\end{table}
\\
由表格可以直接得到,该问题有最优解\\
最优解: $(x_1,x_2)=(24,14)$\\
最优值:$f_{min}=-440$

(2):该问题是最大化问题,转而求相应的最小化问题,化为标准型
\[
\begin{split}
min \quad -(x_1+3x_2)\\
\begin{aligned}
s.t.\quad &2x_1+3x_2+x_3=6,\\
     &-x_1+x_2+x_4=1,\\
     &x_j \geq 0,j=1,\cdots,4.\\
\end{aligned}
\end{split}
\]
初始化单纯形表
\begin{table}[h]
    \setlength{\belowcaptionskip}{0.cm}
    \centering
    \begin{tabular}
        {>{}rccccc}
        \toprule[1pt]
        \rowcolor[gray]{0.9}    &$x_1$ &$x_2$   &$x_3$  &$x_4$ & \\
        \midrule
        $x_3$   &2   &3  &1  &0 & 6  \\
        $x_4$   &-1 &\multicolumn{1}{>{\columncolor{green}[0pt][0pt]}c}{1}    &0  &1 & 1  \\
        & 1 & 3 & 0 & 0 & 0\\
        \bottomrule[1pt]
        \end{tabular}
\end{table}
\\
第二次迭代
\begin{table}[h]
    \setlength{\belowcaptionskip}{0.cm}
    \centering
    \begin{tabular}
        {>{}rccccc}
        \toprule[1pt]
        \rowcolor[gray]{0.9}    &$x_1$ &$x_2$   &$x_3$  &$x_4$ & \\
        \midrule
        $x_3$   &\multicolumn{1}{>{\columncolor{green}[0pt][0pt]}c}{5}   &0  &1  &-3 & 3  \\
        $x_4$   &-1 &1    &0  &1 & 1  \\
        & 1 & 3 & 0 & 0 & 0\\
        \bottomrule[1pt]
        \end{tabular}
\end{table}
\\
\\
\\
\\
\\ \\ \\ \\

第三次迭代
\begin{table}[h]
    \setlength{\belowcaptionskip}{0.cm}
    \centering
    \begin{tabular}
        {>{}rccccc}
        \toprule[1pt]
        \rowcolor[gray]{0.9}    &$x_1$ &$x_2$   &$x_3$  &$x_4$ & \\
        \midrule
        $x_3$   &1   &0  &$\frac{1}{5}$  &-$\frac{3}{5}$ & $\frac{3}{5}$  \\
        $x_4$   &0 &1    &$\frac{1}{5}$  &$\frac{2}{5}$ & $\frac{8}{5}$  \\
        & 0 & 0 & $-\frac{4}{5}$ & -$\frac{3}{5}$ & -$\frac{27}{5}$\\
        \bottomrule[1pt]
        \end{tabular}
\end{table}
\\
由表格可以直接得到,该问题有最优解\\
最优解: $(x_1,x_2)=(\frac{3}{5},\frac{8}{5})$\\
最优值:$f_{max}=\frac{27}{5}$
\end{document}